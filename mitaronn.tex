\documentclass[dvipdfmx,autodetect-engine,11pt]{jsarticle}
\usepackage[utf8]{inputenc}
\usepackage{here}
\usepackage{url}
\usepackage[dvipdfmx]{graphicx}
\urlstyle{rm}
\setcounter{page}{0}
\usepackage{comment}
\usepackage{bxtexlogo}
\usepackage{amsmath}

\setlength{\textheight}{\paperheight}
\setlength{\topmargin}{4.6truemm}
\addtolength{\topmargin}{-\headheight}
\addtolength{\topmargin}{-\headsep}
\addtolength{\textheight}{-60truemm}

\setlength{\textwidth}{\paperwidth}
\setlength{\oddsidemargin}{4.6truemm}
\setlength{\evensidemargin}{4.6truemm}
\addtolength{\textwidth}{-60truemm}

\usepackage{tikz}%お絵描きツール。本番はいらないので必ず削除!!

\title{三田論タイトル\\---副題的な---\footnote{本稿は、2024年11月X日からY日に開催される、三田祭論文コンテストのために作成したものである。本稿の作成にあたっては、廣瀬康生教授(慶應義塾大学)と廣瀬康生研究会11期生(慶應義塾大学)の方々から有益かつ熱心なコメントを頂戴した。ここに記して感謝の意を表したい。しかしながら、本稿にあり得る誤り、主張の一切の責任はいうまでもなく筆者たち個人に帰するものである。}}
\author{柴田俊亮\footnote{慶應義塾大学廣瀬康生研究会12期生} 乃万博太郎\footnote{同上} 古屋雄大\footnote{同上} 又木啓充\footnote{同上}}
\date{2024年11月*日}

\begin{document}
\maketitle

\begin{abstract}
1行目\\
名前はとりあえず50音順にした!怒らないでね。
自我丸出しなわけじゃないから!!\\by柴田\\
インフレなンゴ。かくかくしかじかで助けてクレメンス\\
2段落目\\
【悲報】なんJ民現る\newline

\begin{center}
\begin{tikzpicture}
\node[draw,fill=red!20!white,circle] (S) at (0,0) {廣瀬};
\node[draw,fill=blue!20!white,circle] (G) at (2,2) {康生};
\draw[very thick,->] (S) -- (G);
\end{tikzpicture}
\end{center}


\end{abstract}

\newpage
\tableofcontents%目次作るコマンド
\newpage

\section{はじめに}

ここでは何か語らせてもらうよ。\newline
boxに過去の\LaTeX(要は\TeX)ファイルあったので拝借した。\\
(おそらく)偉大な先人による秘伝継ぎ足しのたれ的なコードと思われる。\\
私たちもその一助になろう←深夜テンションで変なこと語っておる\\


\begin{equation}
i_t = r_t + \pi^e
\end{equation}


\section{手法}
\subsection{モデルの説明}

\begin{equation}
    U_t = E_0\sum_{t=0}^{\infty}\beta^t\exp(z_t^b)[\frac{(C_t - \theta C_{t-1})^{1-\sigma}}{1-\sigma} - (z_t^*)^{1-\sigma}\exp(z_t^h)\int_0^1\frac{(h_t(m)^{1+\chi}}{1 + \chi}dm]
\end{equation}
手打ちは疲れる
\subsection{推計の説明}

\section{推定結果}

\section{政策検証}

\section{結果と考察}

\section{頑健性チェック}

\maketitle%←これ必要なのかわからん
\section{結び}
とりあえす今日はこんな感じで

\maketitle
\addcontentsline{toc}{section}{参考文献}
\begin{thebibliography}{99}
	\bibitem {}廣瀬康生 (2012) 『DSGEモデルによるマクロ実証分析の方法』 三菱経済研究所.
 \end{thebibliography}

\maketitle
\addcontentsline{toc}{section}{Appendix}
\section*{Appendix}


\end{document}